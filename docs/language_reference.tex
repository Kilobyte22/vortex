% !TEX TS-program = pdflatex
% !TEX encoding = UTF-8 Unicode

\documentclass{article}
\usepackage[utf8]{inputenc}
\usepackage[T1]{fontenc}

\usepackage{geometry}
\geometry{a4paper}
% \geometry{margin=2in} % for example, change the margins to 2 inches all round

\usepackage{graphicx} % support the \includegraphics command and options
\usepackage[parfill]{parskip} % Activate to begin paragraphs with an empty line rather than an indent

\usepackage{booktabs} % for much better looking tables
\usepackage{array} % for better arrays (eg matrices) in maths
\usepackage{paralist} % very flexible & customisable lists (eg. enumerate/itemize, etc.)
\usepackage{verbatim} % adds environment for commenting out blocks of text & for better verbatim
\usepackage{subfig} % make it possible to include more than one captioned figure/table in a single float
% These packages are all incorporated in the memoir class to one degree or another...
\usepackage{textcomp}
\usepackage{hyperref}

%%% HEADERS & FOOTERS
\usepackage{fancyhdr} % This should be set AFTER setting up the page geometry
\pagestyle{fancy} % options: empty , plain , fancy
\renewcommand{\headrulewidth}{0pt} % customise the layout...
\lhead{}\chead{}\rhead{}
\lfoot{}\cfoot{\thepage}\rfoot{}

%%% SECTION TITLE APPEARANCE
\usepackage{sectsty}
\allsectionsfont{\sffamily\mdseries\upshape} % (See the fntguide.pdf for font help)
% (This matches ConTeXt defaults)

%%% ToC (table of contents) APPEARANCE
\usepackage[nottoc,notlof,notlot]{tocbibind} % Put the bibliography in the ToC
\usepackage[titles,subfigure]{tocloft} % Alter the style of the Table of Contents
\renewcommand{\cftsecfont}{\rmfamily\mdseries\upshape}
\renewcommand{\cftsecpagefont}{\rmfamily\mdseries\upshape} % No bold!

\usepackage{csquotes}
\usepackage{listings}
\usepackage{syntax}
\usepackage{color}

\definecolor{keyword1}{RGB}{0,102,153}
\definecolor{keyword2}{RGB}{0,153,102}
\definecolor{keyword3}{RGB}{0,153,255}
\definecolor{comment}{RGB}{204,0,0}
\definecolor{function}{RGB}{153,102,255}
\definecolor{digit}{RGB}{255,0,0}
\definecolor{string}{RGB}{255,0,204}
\definecolor{rule}{RGB}{192,192,192}
\definecolor{back}{RGB}{250,250,250}

\usepackage[linecolor=rule,
    backgroundcolor=back,innertopmargin=12pt,skipbelow=5pt]{mdframed}

\newenvironment{bnf}
{
\begin{mdframed}
\begin{grammar}
}
{
\end{grammar}
\end{mdframed}
}

\lstdefinelanguage{vortex}{
  keywordstyle={\color{keyword2}\bfseries},
  keywordstyle=[2]{\color{keyword1}\bfseries},
  keywordstyle=[3]{\color{keyword2}\bfseries},
  keywordstyle=[4]{\color{keyword3}\bfseries},
  keywords={},
  morekeywords=[2]{and,as,asr,band,bnot,bor,break,bsl,bsr,bxor,case,cfn,clone,coro,cycle,do,
    else,end,enum,fn,for,goto,if,in,let,loop,macro,match,module,new,not,or,quote,redo,result,
    return,seq,set,unquote,when,while,with,yield},
  morekeywords=[3]{glob,rec,assert,self,super},
  morekeywords=[4]{__FILE__,__LINE__,false,nil,true},
  otherkeywords={->},
  morecomment=[l][\color{comment}]{//},
  morecomment=[s][\color{comment}]{/*}{*/},
  stringstyle=\color{string},
  morestring=[b]",
  morestring=[b]',
  morestring=[b]""",
  morestring=[b]''',
  showstringspaces=false,
  frame=single,
  rulecolor=\color{rule},
  backgroundcolor=\color{back},
  numbers=left
}

\lstdefinelanguage{text}{
  frame=single,
  rulecolor=\color{rule},
  backgroundcolor=\color{back}
}

%%% END Article customizations

%%% The "real" document content comes below...

\title{The Vortex programming language}
\author{Daniel "q66" Kolesa}
%\date{} % Activate to display a given date or no date (if empty),
         % otherwise the current date is printed 

\begin{document}
\maketitle
\thispagestyle{empty}

\abstract
Vortex is a scripting language created to explore the possibilities of using
Lua as an intermediate language. While compiling to performant Lua (comparable
to handwritten code), Vortex tries to simplify programming by introducing new
features (such as lambda expressions, macros, lists and objects) that Lua
lacked before (while not introducing anything that is not present in regular
Lua). Vortex gains inspiration from well known programming languages mainly
in the functional paradigm, such as OCaml, F\#, Scheme and Rust besides Lua
itself. The language itself tries to offer the programmer multiple paradigms
-- functional, procedural, object oriented (prototype based -- delegative with
    multiple inheritance) and metaprogramming. Vortex does not try to create a
\enquote{preprocessor} for Lua (akin to attempts like CoffeeScript/MoonScript).

\newpage
\thispagestyle{empty}
\tableofcontents
\pagebreak
%\setcounter{page}{1}

\section{Introduction}
This document attempts to create a reference manual for the Vortex programming
language. It tries to cover:
\begin{itemize}
  \item Various aspects of the language, including lexical analysis and all of
      the language's structures. The standard library is not covered (and at
      the point of writing, not yet designed).
  \item Rationale for many of the language features.
  \item Examples of usage.
\end{itemize}
It tries to be just documentation, not a tutorial. For a tutorial, check out
other sources (such as the Vortex wiki).
\subsection{Disclaimer}
Vortex is still in heavy development. This document will change as the language
changes -- do not use it in production. While the basic idea and style is
given, language features may appear, change and vanish. The implementation
may not reflect the actual state of the language. Some features described here
are not yet implemented at the time of writing. Unimplemented features will be
clearly marked as such in their section headers.
\subsection{Conventions}
The formal grammar of Vortex specified here is written using an extended
dialect of BNF. The BNF strings may contain \verb|/regexes/|. Example code is
written using \verb|monospace| blocks. Character ranges may be used for obvious
things, such as \verb|a-Z| or \verb|0-9|.

\section{Concepts}
Vortex shares most of the basic concepts with Lua.
\subsection{Variables and values}
Vortex is an untyped language, or unityped to be precise -- there is a single
data type in the language, a value, similar to Lua, Python, JavaScript and
so on. The values are tagged. The variable holds the data type and its value
holds the tag.\\
Vortex type tags are shared with Lua, being represented by \verb|nil|,
\verb|boolean|, \verb|number|, \verb|string|, \verb|function|,
\verb|userdata|, \verb|thread| and \verb|table|. The \verb|nil| tag can have
a single value, \verb|nil|. The \verb|boolean| tag can have the \verb|true|
and \verb|false| values. The only two values that evaluate as false in
conditions are \verb|false| and \verb|nil|. The tag \verb|number| represents
a floating point number (by default double precision), depending on the
underlying C. The tag \verb|string| is an immutable sequence of bytes.
Vortex is like Lua 8-bit clean. Strings can contain any byte, including
embedded zeros.\\
The tag \verb|userdata| has the same meaning as in Lua. The managed kind
represents a managed (garbage collected) block of memory that can have
a metatable. The light kind represents a raw pointer. The tag \verb|thread|
is represented by coroutines which are identical with Lua's. Functions are
represented using the tag \verb|function|. They're transparent closures.\\
And finally \verb|table| is the ultimate do-it-all data structure of Vortex.
It represents an associative array (unordered hash map). It's heterogenous,
being able to hold any arbitrary value except nil in both key and value
positions.\\
Vortex also has lists. These are similar to lists in Lisp or other functional
languages. They don't hold their own tag -- they're tables with specific
behavior.\\
Functions, threads, full userdata and tables are always accessed by reference,
everything else by value (copied). All values are first class values. You can
store them, pass them around and return from functions.\\
There are two types of variables in Vortex, locals and table fields. Global
variables are merely fields of the global table called \verb|_G|. Local and
global variables have to be declared before they can be assigned or used,
which is different from Lua, where assignment to undeclared variable creates
a global and reading an undeclared variable results in the \verb|nil| value.
You declare and define variables using the \verb|let| expression.
\subsection{Metaprogramming}
Vortex features two types of metaprogramming -- static and dynamic. Dynamic
metaprogramming is identical to Lua's and is represented by reflection over
tables and by metatables. Vortex however also introduces static metaprogramming
(performed at compile time). That is represented by macros. Macros work at
AST level. They're hygienic -- identifiers inside them cannot escape unless
explicitly desired.\\
As Vortex's module system is dynamic, evaluated already past the macro
expansion step, there is a system of language \enquote{extensions} that
allow macros across modules.
\subsection{Modules}
As said above, Vortex features a dynamic module system. It's taken from
Lua itself and it's compatible. It's further documented in the standard
library part.
\subsection{Tables and metatables}
As said before, the table is the ultimate do-it-all and the only data structure
in Vortex and Lua. Vortex also has lists, but they're simply tables with
syntax. The real power of tables comes from metatables.\\
Every table can have a metatable, or any value actually. In case of tables
and userdata, each piece of userdata and each table can have (but doesn't
have to have) its own, unique metatable. With other values, the metatable
is specific to the type tag (for example all strings share a single
metatable). Of course, you can share metatables between tables, but
it's all explicit.\\
Metatables are ordinary tables. A metatable contains metamethods. Metamethods
define fallbacks for operations on the value. What does this mean? It means
you can change the semantics on tables and other values and extend the
language this way. For example, a metamethod \verb|__add| defines
what happens when you add the value together with some other value.\\
Metamethods have intuitive naming scheme based on the name of the event
they handle. They're prefixed with two underscores. You can get a metatable
of a value using the \verb|get_mt| function. You can set a metatable using
\verb|set_mt|.
\subsection{Coroutines}
Lua offers coroutines and so does Vortex. Coroutines are used to do
collaborative multithreading. A coroutine is a thread -- it has its own
stack, but it's not an OS thread. Unlike Lua, Vortex offers specialized
coroutine syntax.\\
A coroutine is somewhat similar to a function. When you create a coroutine
from a function, you get a thread object. The thread object is by default in
a suspended state. When you resume the thread object, the function runs. If
it's a regular function you could call normally, the thread object dies and
can't be resumed again.\\
There is a language construct in Vortex called \verb|yield|. Yielding a
coroutine results in the thread object being suspended at the point of
yield. You can then again resume the coroutine and it'll run until the
next yield.\\
You can pass one or more expressions to \verb|yield|. The grammar is defined
in the language section. The \verb|resume| function will return values
of the expressions passed to \verb|yield|. It also returns the return
value of the function when the thread object dies. \verb|Yield| thus has
the same semantics as \verb|return| when it comes to resuming, except that
it only suspends the coroutine instead of making it die.\\
You can also pass expressions to \verb|resume|. They'll be returned inside
the coroutine by the yield. An example:
\begin{lstlisting}[language=vortex]
// special syntax for coroutines provided by Vortex
// coroutines can also be created from functions using
// special calls with the same effect
let my_coroutine = coro a, b do
    print("AB", a, b)
    let (c, d) = yield (5, 10)
    print("CD", c, d)
    return (15, 20)
end
print(resume(my_coroutine, 2, 3))
print(resume(my_coroutine, 6, 7))
print(resume(my_coroutine, 8, 9))
\end{lstlisting}
This will print something like
\begin{lstlisting}[language=text]
AB 2 3
true 5 10
CD 6 7
true 15 20
false cannot resume dead coroutine
\end{lstlisting}
What exactly happened here? First, we resumed the coroutine, passing values
\verb|2| and \verb|3| to it. Here it was like a function call. The passed
values became \verb|a| and \verb|b|. The coroutine then printed them and
yielded -- suspended itself, passing back values \verb|5| and \verb|10|
and expecting \verb|c| and \verb|d| for the next resume. The \verb|resume|
function then printed true (\verb|resume| returns the state the coroutine
was in -- whether it was dead or alive as the first result) plus the
yielded values.\\
Then we resumed once again. The values passed to \verb|resume| became
\verb|c| and \verb|d| as expected. The coroutine printed the values and
returned more values. The coroutine has died by now. The \verb|resume| call
once again returned \verb|true| (because it was alive at the point of
execution) and the return values. A third resume returned \verb|false|,
because the coroutine was already dead, plus an error message.\\
That's coroutines in a nutshell. There are two variants of coroutines
in Vortex, regular coroutines and generators. There's a subtle difference
in the amount of power they give you and in the simplicity of writing.
Please refer to the later sections.
\section{Lexical}
Vortex utilizes a fully free form syntax. Whitespace of any form is ignored,
used only to delimit tokens where it would otherwise be ambiguous.
\subsection{Encoding}
\begin{bnf}
<alpha> ::= \lit{a-zA-Z}

<digit> ::= \lit{0-9}

<hexdigit> ::= \lit{0-9a-fA-F}
\end{bnf}
The input is a sequence of bytes. Vortex does not handle Unicode in any way,
but it's UTF-8 clean. All of the grammar is confined to the ASCII. Because
it's UTF-8 clean, Unicode strings and such are allowed and passed to the
output without any processing. If an UTF-8 BOM is found, it's skipped
automatically. If a shebang line is found in the beginning, it's skipped
as well.
\subsection{Whitespace}
\begin{bnf}
<whitespace> ::= \lit{ }
    \alt \lit{\\n}
    \alt \lit{\\r}
    \alt \lit{\\t}
    \alt \lit{\\f}
    \alt \lit{\\v}
\end{bnf}
The Vortex lexer does not care about whitespace for anything else than token
separation. Whitespace is not needed to separate tokens; the source is simply
read character by character and when a token ending is found, it just goes on
to the next one. The BNF above shows all possible forms of whitespace in
Vortex. As you can see, newlines are treated as regular whitespace as well
(with the exception of incrementing the line number). Newline and carriage
return can be used in pairs no matter what order (handles the common cases
'\textbackslash n' and '\textbackslash r\textbackslash n' plus
'\textbackslash n\textbackslash r' for rare platforms).
\subsection{Comments}
% ugly textcolor hack to make some editors happy with escaped $
\begin{bnf}
<comment> ::= \lit{//} /.*\$/
    \alt \lit{/*} \{ /./ | <comment> \} \lit{*/}
\end{bnf}
Vortex uses C(++) style comments. Short comments comment out everything until
the end of the line. Long comment can span multiple lines and are enclosed
between delimiters. Unlike C(++), Vortex allows for nesting of comments
(thus it requires you to keep them balanced). Comments do not get past
lexical analysis.
\subsection{Identifiers}
\begin{bnf}
<ident> ::= (\lit{_} | <alpha>) \{ \lit{_} | <alpha> | <digit> \}

<identlist> ::= <ident> \{ \lit{,} <ident> \}
\end{bnf}
Vortex identifiers (names) can consist of alphanumeric characters (ASCII only)
and underscores. They can't start with a digit, but a digit can be present
anywhere else in the identifier. Identifiers starting with an underscore
followed by an uppercase character are reserved. This is by convention and
not enforced by the compiler.
\subsection{Keywords}
\begin{bnf}
<opkeyword> ::= \lit{band}
    \alt \lit{bor}
    \alt \lit{bxor}
    \alt \lit{asr}
    \alt \lit{bsr}
    \alt \lit{bsl}
    \alt \lit{and}
    \alt \lit{or}

<opkeywordass> ::= \lit{band=}
    \alt \lit{bor=}
    \alt \lit{bxor=}
    \alt \lit{asr=}
    \alt \lit{bsr=}
    \alt \lit{bsl=}

<keyword> ::= \lit{as}
    \alt \lit{break}
    \alt \lit{case}
    \alt \lit{cfn}
    \alt \lit{clone}
    \alt \lit{coro}
    \alt \lit{cycle}
    \alt \lit{do}
    \alt \lit{else}
    \alt \lit{end}
    \alt \lit{enum}
    \alt \lit{false}
    \alt \lit{fn}
    \alt \lit{for}
    \alt \lit{glob}
    \alt \lit{goto}
    \alt \lit{if}
    \alt \lit{in}
    \alt \lit{let}
    \alt \lit{loop}
    \alt \lit{macro}
    \alt \lit{match}
    \alt \lit{module}
    \alt \lit{new}
    \alt \lit{nil}
    \alt \lit{quote}
    \alt \lit{rec}
    \alt \lit{redo}
    \alt \lit{result}
    \alt \lit{return}
    \alt \lit{seq}
    \alt \lit{set}
    \alt \lit{true}
    \alt \lit{unquote}
    \alt \lit{when}
    \alt \lit{while}
    \alt \lit{with}
    \alt \lit{yield}
    \alt \lit{__FILE__}
    \alt \lit{__LINE__}
    \alt <opkeyword>
    \alt <opkeywordass>
    \alt <unopkeyword>

<unopkeyword> ::= \lit{not}
    \alt \lit{bnot}
\end{bnf}
These are \enquote{reserved words} in Vortex. They cannot be used as
identifiers. Some of these specified in \verb|opkeyword| have an assignment
form. Vortex is case sensitive, the keywords only are keywords in lowercase.
For example, \verb|match| is a keyword but \verb|MATCH| or \verb|Match| can
be used as valid identifiers.
\subsection{Other tokens}
\begin{bnf}
<binop> ::= \lit{=}
    \alt \lit{==}
    \alt \lit{\textgreater}
    \alt \lit{\textgreater=}
    \alt \lit{\textless}
    \alt \lit{\textless=}
    \alt \lit{!=}
    \alt \lit{\%}
    \alt \lit{~}
    \alt \lit{+}
    \alt \lit{++}
    \alt \lit{*}
    \alt \lit{**}
    \alt \lit{-}
    \alt \lit{/}
    \alt \lit{::}
    \alt <opkeyword>

<assop> ::= \lit{~=}
    \alt \lit{+=}
    \alt \lit{++=}
    \alt \lit{*=}
    \alt \lit{**=}
    \alt \lit{-=}
    \alt \lit{/=}
    \alt \lit{::=}
    \alt \lit{\%=}
    \alt <opkeywordass>

<unop> ::= \lit{-}
    \alt \lit{\#}
    \alt <unopkeyword>

<othertok> ::= \lit{(}
    \alt \lit{)}
    \alt \lit{->}
    \alt \lit{.}
    \alt \lit{..}
    \alt \lit{...}
    \alt \lit{;}
    \alt \lit{;;}
    \alt \lit{\$}
    \alt \lit{\$(}
\end{bnf}
Here you can see all the other tokens used by Vortex with separate categories
for binary, unary and assignment operators.
\subsection{Number literals}
\begin{bnf}
<hexnum> ::= (\lit{0x} | \lit{0X}) \{ <hexdigit> \} [ \lit{.} ]
    \{ <hexdigit> \} [ (\lit{p} | \lit{P}) [ \lit{+} | \lit{-} ] <digit> ]

<decnum> ::= \{ <digit> \} [ \lit{.} ] \{ <digit> \} [ (\lit{e} | \lit{E})
    [ \lit{+} | \lit{-} ] <digit> ]

<numliteral> ::= <hexnum>
    \alt <decnum>
\end{bnf}
Vortex does not make a difference between integers and floats. Numbers follow
the Lua format. Hexadecimal numbers start with \verb|0x| or \verb|0X| (the
former is better). In numbers like \verb|0.6| you can omit the zero and write
just \verb|.6|. Hexadecimal constants behave similarly. Optional decimal
exponent is marked with \verb|e| or \verb|E| (binary exponent in hex constants
is \verb|p| or \verb|P|).
\subsection{String literals}
\begin{bnf}
<strliteral> ::= <strprefix> (<strlong> | <strshort>)

<strprefix> ::= /[eErR]*/

<strshort> ::= \lit{\textquotedbl} <strshortelem> \lit{\textquotedbl}
    \alt \lit{\textquotesingle} <strshortelem> \lit{\textquotesingle}

<strlong> ::= \lit{\textquotedbl\textquotedbl\textquotedbl} <strlongelem>
        \lit{\textquotedbl\textquotedbl\textquotedbl}
    \alt \lit{\textquotesingle\textquotesingle\textquotesingle} <strlongelem>
        \lit{\textquotesingle\textquotesingle\textquotesingle}

<strshortelem> ::= ? short string contents ?
    \alt <stresc>

<strlongelem> ::= ? long string contents ?
    \alt <stresc>

<stresc> ::= \lit{\\a}
    \alt \lit{\\b}
    \alt \lit{\\f}
    \alt \lit{\\n}
    \alt \lit{\\r}
    \alt \lit{\\t}
    \alt \lit{\\v}
    \alt \lit{\\z}
    \alt \lit{\\\textquotedbl}
    \alt \lit{\\\textquotesingle}
    \alt \lit{\\}
\end{bnf}
String literals in Vortex are similar to Python's. They are represented by the
\verb|<string>| token in the final stream. There are two types of string
literals, short and long literals.\\
Short literals are delimited with either single or double quotes and
typically hold a single line. They interpret escape sequences. A backslash at
the end of the line can be used to make them span multiple lines. You need to
escape nested single or double quotes depending on the used delimiter (e.g.
single quote delimited strings need to escape nested single quotes but
not double quotes).\\
Long literals behave similarly. They're delimited by three repeated either
single or double quotes. Escape sequences work the same, and long literals can
span multiple lines without backslashes. You don't need to escape quotes except
when three subsequent quotes are used (because otherwise they'd terminate the
string).\\
You can prefix both types of literals with either \verb|e| or \verb|r| (and
uppercase versions -- no difference there). The former enables interpolation
on that string -- it will interpret \verb|$var| and \verb|$(expr)| in the
string, where \verb|var| is any Vortex variable you can access at that point
and \verb|expr| is any Vortex expression. For example
\begin{lstlisting}[language=vortex]
for k, v in pairs([ 5, 10, 15 ]) -> print(e"$k -> $(v + 2)")
\end{lstlisting}
will print
\begin{lstlisting}[language=text]
1 -> 7
2 -> 12
3 -> 17
\end{lstlisting}
Interpolated strings allow you to escape the dollar sign to prevent
interpolation.
\\
The latter prefix turns the strings into raw strings. That means no escape
sequences are interpreted and instead they apparear in the result verbatim.
If you escape quotes, the backslashes will be visible in the string. The same
applies about backslashes used to escape newlines in short literals.\\
There is one non-standard escape sequence, \verb|\z|. It skips the following
span of whitespace characters (including newlines) in both short and long
literals. That is useful to break a short string literal into multiple lines
and indent the lines without actually including the newlines and indentation
in the string.\\
You can also insert an arbitrary byte in the string. That can be done either
with an escape sequence \verb|\xXX|, where \verb|XX| is a sequence of two
hexadecimal digits (e.g. \verb|\x4F| for uppercase O) or with an escape
sequence \verb|\ddd| where \verb|ddd| is a sequence of up to three decimal
digits (\verb|\79| would be the uppercase O).

\section{Expressions}
\begin{bnf}
<exp> ::= <statexp>

<explist> ::= [ <exp> \{ \lit{,} <exp> \} ]
\end{bnf}
Vortex is a language that consists purely of expressions. No statements are
present in the language in that sense an expression of any type can appear
in any context. One exception is block scopes. Only expressions that can
cause some sort of side effect can appear there (e.g. variable assignment,
function call and so on, in the BNF they're called \verb|statexp|).
The rationale for this is that doing it otherwise doesn't really make sense
-- allowing inclusion of arbitrary expressions in blocks would result in code
doing nothing, and that's better filtered out at compile time.
\subsection{Main scope}
\begin{bnf}
<chunk> ::= \{ <statexp> [ \lit{;} ] \} [ (<retexp> | <resexp>) [ \lit{;} ] ]

<mainchunk> ::= \{ (<statexp> | <macro>) [ \lit{;} ] \}
    [ (<retexp> | <resexp>) [ \lit{;} ] ]
\end{bnf}
The parsing of Vortex begins in the main scope. The main scope is a chunk.
A chunk is a sequence of side effect based expressions optionally separated
with semicolons. It may be ended with either a \verb|return| or a \verb|result|
expression. Having another expression after one of these results in syntax
error.\\
The main chunk can unlike any other chunk define macros, which are described
in their own section later.
\subsection{Blocks}
\begin{bnf}
<blockend> ::= \lit{end} | \lit{;;}

<block> ::= \lit{do} <chunk> <blockend>

<expscope> ::= \lit{->} <exp> | <block>

<statscope> ::= \lit{->} <statexp> | <block>

<statexp> ::= <block>
\end{bnf}
Blocks represent chains of expressions. A block consists of a \verb|do|
keyword, a chunk (see above) and either an \verb|end| keyword or two
semicolons. The semicolon ending is useful in inline blocks or in
Lisp style formatted code. Blocks themselves are expressions. The \verb|result|
expression can be used to specify their value.
\subsection{Return and result expressions}
\begin{bnf}
<exporlist> ::= (<exp> | \lit{(} <explist> \lit{)})

<retexp> ::= \lit{return} <exporlist>

<resexp> ::= \lit{result} <exporlist>

<statexp> ::= <retexp> <resexp>
\end{bnf}
These two expressions are used to manipulate values. The \verb|return|
expression jumps out of a function, making it return value(s) (in the main
scope it makes the module return a value later used with \verb|require|).\\
The \verb|result| expression is very similar at first. It however works on
scope level -- it basically sets a value the current block will evaluate to.
With regular functions, this is pretty much the same (as a block is typically
a function return value) but you can see the difference when working with
expression blocks. For example:
\begin{lstlisting}[language=vortex]
// this function returns 7, x is 5.
fn foo() do
    let x = do
        result 5
    end
    return x + 2
end

// this function returns 5, never reaching the final return.
fn bar() do
    let x = do
        return 5
    end
    return x + 2
end
\end{lstlisting}
\subsection{Binary and unary expressions}
\begin{bnf}
<assexp> ::= (<ident> | <indexp>) <assop> <exp>

<binexp> ::= <exp> <binop> <exp>

<unexp> ::= <unop> <exp>

<exp> ::= <binexp>
    \alt <unexp>

<statexp> ::= <assexp>
\end{bnf}
Binary expressions are expressions that consist of two operands and an
operator. The operator is in infix form. The BNF here does not describe
operator precedences. Assignment operators are treated differently as
it's required to ensure that the left operand is an lvalue (you can't
assign an arbitrary expression). Assignment expressions can also be used in
statement form, unlike any other binary expression.\\
Unary expressions are expressions that consist of an operand and an operator
in prefix form.
Here you can see operator precedences for binary and unary operators in Vortex.
\begin{center}
\begin{tabular}{| l | l | l |}
\hline
\textbf{Operator} & \textbf{Precedence} & \textbf{Associativity} \\ \hline
=, +=, -=, *=, /=, \%=, \~{}=, ++=, ::=, **=, & 1 & right \\
band=, bor=, bxor=, asr=, bsr=, bsl= & & \\ \hline
or & 2 & left \\ \hline
and & 3 & left \\ \hline
==, != & 4 & left \\ \hline
\textless, \textless=, \textgreater, \textgreater= & 5 & left \\ \hline
\~{} & 6 & right \\ \hline
bor & 7 & left \\ \hline
bxor & 8 & left \\ \hline
band & 9 & left \\ \hline
asr, bsr, bsl & 10 & left \\ \hline
+, - & 11 & left \\ \hline
*, /, \% & 12 & left \\ \hline
++ & 13 & left \\ \hline
:: & 14 & right \\ \hline
-, not, \#, bnot & 15 & unary \\ \hline
** & 16 & right \\ \hline
\end{tabular}
\end{center}
Most of the operators should be clear when it comes to meaning. Assignment
operators in form \verb|lhs op= rhs| are equivalent to \verb|lhs = lhs op rhs|.
\verb|~| means concatenation, \verb|++| is a join operator (for tables),
\verb|::| is a cons operator (as in Lisp), \verb|#| retrieves the length
of the given expression, \verb|**| raises lhs to a power of rhs. \verb|band|,
\verb|bor|, \verb|bxor|, \verb|asr|, \verb|bsr|, \verb|bsl| mean bitwise AND,
bitwise OR, bitwise XOR, arithmetic right shift, bitwise right shift and
bitwise left shift respectively. The rest of the operators is functionally
equivalent to those in C.
\subsection{Tables and lists}
\begin{bnf}
<tableitem> ::= [ ([ \lit{\$} ] <ident> | \lit{\$(} <exp> \lit{)})
    \lit{:} ] <exp>

<listitem> ::= <exp>

<tableexp> ::= \lit{\{} [ <tableitem> \{ \lit{,} <tableitem> \} ] \lit{\}}

<listexp> ::= \lit{[} <listitem> \{ \lit{,} <listitem> \} ] \lit{]}

<exp> ::= <tableexp>
    \alt <listexp>
\end{bnf}
Vortex provides two types of built-in data structures. They're tables and
lists.\\
Tables work in the same way as in Lua. They are something between an array,
an associative array (unordered hash map) and an object. Array and hash
elements can be both present in a single table. Tables are already described
in their own section above.\\
Syntactically, table literals are enclosed in curly braces. If you want a
value that has a key, you use the colon syntax. For example:
\begin{lstlisting}[language=vortex]
let array = { 5, 10, 15 } // an array of 3 elements
assert(array[1] == 5 and array[3] == 15)

// contains keys "foo", "abcd", "bar"
let assarray = { foo: "bar", $("ab" ~ "cd"): "baz", $bar: "xyz" }

// combined
let comb = { 5, 10, hello: "world", 15 } 
\end{lstlisting}
As you can see, keys can be arbitrary expressions. Keys can be any value
except \verb|nil|.\\
Arrays count from 1. They're basically associative arrays with keys that
are numbers, however, Lua optimizes this by storing array elements in their
own section. You can assign to a table as you need. Array length is retrieved
using the \verb|#| operator. For example, this way you can append:
\begin{lstlisting}[language=vortex]
let array = { 5, 10, 15 }
array[#array + 1] = 20
\end{lstlisting}
Tables can have metatables as mentioned in the section above.\\
Lists are another data structure of Vortex. They use square brackets. Vortex
lists are singly linked lists in concept similar to Lisp lists. You can
construct a list multiple ways:
\begin{lstlisting}[language=vortex]
// using the list syntax
let lst = [ 5, 10, 15, 20 ]
// using the cons binary operator
let tsl = 5 :: 10 :: 15 :: 20 :: nil
\end{lstlisting}
Both ways are equivalent. A list consists of the \enquote{head} element
and the \enquote{tail} element. Here, the first head is 5, the tail is another
list where the head is 10, the tail is another list, the head again is 25,
followed by another list where the head is 20 and the tail is \verb|nil|.\\
Sometimes lists are more efficient than just tables. It depends on the use.
Lists are implemented using tables in the runtime.
\subsection{Primary expressions}
\begin{bnf}
<primaryexp> ::= \lit{(} <exp> \lit{)}
    \alt <tableexp>
    \alt <listexp>
    \alt \lit{\$(} <exp> \lit{)}
    \alt \lit{\$} <ident>
    \alt <ident> [ \lit{!} \lit{(} <explist> \lit{)} ]
    \alt <numliteral>
    \alt <strliteral>
    \alt \lit{nil}
    \alt \lit{true}
    \alt \lit{false}

<exp> ::= <primaryexp>
\end{bnf}
Primary expressions are simple expressions that can have a suffix. A suffix
is for example a parameter list (a call), brackets with an expression
(indexing) or a dot (simple indexing). Primary expressions typically don't
have a side effect, thus they can't be used in statement form unless postfixed
as a call. You can wrap any arbitrary expression in parens to get a primary
expression. All types of simple literals as well as table and list literals
and macro expansions are primary expressions.
\subsection{Suffixed expressions}
\begin{bnf}
<fcallsuffix> ::= \lit{(} <explist> \lit{)}
    \alt <tableexp>
    \alt <strliteral>

<mcallsuffix> ::= \lit{:} <ident> <fcallsuffix>

<tcallsuffix> ::= <fcallsuffix>
    \alt <mcallsuffix>

<indexsuffix> ::= \lit{.} <ident>
    \alt \lit{[} <exp> \lit{]}

<expsuffix> ::= [ <expsuffix> ] (<indexsuffix> | <tcallsuffix>)

<suffixedexp> ::= <primaryexp> <expsuffix>

<exp> ::= <suffixedexp>

<statexp> ::= <suffixedexp> <tcallsuffix>
\end{bnf}
Suffixed expressions are primary expressions with a suffix. A suffix
represents either a call or indexing. Suffixes can be chained. In statement
form, only calls are allowed (not indexing alone). Indexing can be represented
in two forms. The simpler form consists of an expression, a dot and a name.
The name must be a valid indentifier. The more comprehensive form consists
of an expression and an index enclosed in brackets (the index can be an
arbitrary expression). Writing \verb|foo.bar| is equivalent to
\verb|foo["bar"]| assuming \verb|"bar"| contains no non-identifier
characters.\\
Calls can be either method calls \verb|obj:mname(args)| or regular
calls \verb|funcname(args)|. The former is just syntactic sugar for
\verb|obj.mname(obj, args)|. If the sole argument is a table or string literal,
you can omit the parens. That means writing \verb|funcname "foo"| and
\verb|funcname { 5, 10, 15 }| is equivalent to \verb|funcname("foo")| and
\verb|funcname({ 5, 10, 15 })| respectively.
\subsection{Let expression}
\begin{bnf}
<lettype> ::= \lit{rec}
    \alt \lit{glob}

<letexp> ::= \lit{let} [ <lettype> ] (<pattern> | \lit{(} <patternlist>
    \lit{)}) \lit{=} <exporlist>

<statexp> ::= <letexp>
\end{bnf}
The \verb|let| expression provides means to declare and define variables.
It consists of the keyword, a pattern or a pattern list enclosed in parens,
an assignment operator and either a single expression or an expression
list enclosed in parens.\\
You can also optionally provide modifier after the keyword. The modifier
can currently be either \verb|rec| or \verb|glob|. The former is best used
with functions -- it makes it possible for a variable to access itself and
that way you can define recursive functions. Normally functions can't access
themselves. The latter is used to define a global variable -- by default,
all Vortex variables are local.\\
You can't normally declare a variable without definition. However, you can
assign nil to it, which is the same in meaning.\\
For patterns, look up the section about pattern matching. Note that only
patterns that never fail to match can be used with \verb|let|. For convenience,
the table pattern always matches in let but doesn't have to in regular
pattern matching (that is because in let you sometimes want to extract
just a few elements of an array, but in pattern matching you want it
precise).\\
An example of let expression:
\begin{lstlisting}[language=vortex]
let x = 5; // local variable x
let glob y = 10; // global variable y
fn foo() -> [ 5, 10, 15 ]
let [ a, b ] = foo() // a is 5, b is 10 - pattern usage
fn rec bar() -> bar() // recursive
\end{lstlisting}
\subsection{Set expression}
\begin{bnf}
<setexp> ::= \lit{set} (<suffixedexp> | \lit{(} <explist> \lit{)})
<assop> <exporlist>

<statexp> ::= <setexp>
\end{bnf}
The \verb|set| expression assigns values to variables. To assign a single
variable, you can use the regular assignment binary expression. The \verb|set|
expression is used to set multiple variables at once. Like an assignment
expression, the expressions on the left have to be lvalues. This expression
looks quite similar to the \verb|let| expression.\\
The main benefit of setting multiple variables at once is for e.g. swapping
values. Consider this:
\begin{lstlisting}[language=vortex]
let a = 5
let b = 10
// now let's swap a and b
let tmp = a
a = b
b = tmp
\end{lstlisting}
This doesn't look too good. Isn't there a better way? Of course there is.
\begin{lstlisting}[language=vortex]
let a = 5
let b = 10
// and now let's swap
set (a, b) = (b, a)
\end{lstlisting}
Much better, right? The \verb|set| expression evaluates to its left side
after assignment.
\subsection{With expression}
\begin{bnf}
<withexp> ::= \lit{with} (<pattern> | \lit{(} <patternlist> \lit{)})
    \lit{=} <exporlist> <expscope>

<withstat> ::= \lit{with} (<pattern> | \lit{(} <patternlist> \lit{)})
    \lit{=} <exporlist> <statscope>

<statexp> ::= <withstat>

<exp> ::= <withexp>
\end{bnf}
This expression is sort of similar to \verb|let|. It represents a scope-bound
variable. Unlike \verb|let|, it doesn't evaluate to its left side, but instead
to its expression. Consider this:
\begin{lstlisting}[language=vortex]
let x = with y = a() do
    print("hello world!")
    result 5
end
\end{lstlisting}
Here, the variable \verb|y| is bound to the \verb|with| expression scope, being
invisible from anywhere else. The variable \verb|x| will be \verb|5|. Note that
the scope will always evaluate, even when \verb|y| is \verb|nil|.
\subsection{Functions}
\begin{bnf}
<defarglist> ::= <ident> \lit{=} <exp> \{ \lit{,} <ident> \lit{=} <exp> \}

<arglist> ::= [ <identlist> ] [ <defarglist> ] [ [ <ident> ] \lit{...} ]

<fnliteral> ::= \lit{fn} (<arglist> | \lit{(} <arglist> \lit{)}) <expscope>

<fndef> ::= \lit{fn} ([ <lettype> ] <ident> | <ident> [ \lit{.} <ident> ])
    \lit{(} <arglist> \lit{)} <expscope>

<statexp> ::= <fndef>

<exp> ::= <fnliteral>
\end{bnf}
Vortex features first class functions. That means a function can be treated
as a first class citizen -- you can pass it as an argument or return it from
another function. Like in Lua, functions are passed by reference -- you never
access the function value directly, you only access its reference.\\
You can create a function two ways. The first way is an anonymous function.
Because anonymous functions don't have names, you need to assign it to a
variable. It looks like this:
\begin{lstlisting}[language=vortex]
let add = fn a, b -> a + b
print(add(5, 10))
\end{lstlisting}
You can optionally put parens around the argument list. The second way is
a named function. Given the previous example, you can rewrite your function
as:
\begin{lstlisting}[language=vortex]
fn add(a, b) -> a + b
\end{lstlisting}
Named functions can be defined as table members. For example:
\begin{lstlisting}[language=vortex]
let tbl = {}
fn tbl.foo() -> "hello world"
\end{lstlisting}
They can also have modifiers, the same ones as the \verb|let| expression
(unless defined as table members -- then it doesn't make sense). Named
function definitions follow the same rules as the \verb|let| expression.
By default, named functions are local.
\begin{lstlisting}[language=vortex]
let foo = fn -> foo() // won't work - non-recursive
let rec bar = fn -> bar() // works - bar previously declared
fn foo() -> foo() // won't work
fn rec bar() -> bar() // works
\end{lstlisting}
Named functions can be used as statements. Anonymous functions can't.\\
Now, if you look at the examples, you can see they're pretty much lambda
expressions. They take inputs and they return the value specified after
the arrow. You can combine that with blocks. Both \verb|return| and
\verb|result| can be used to specify the function return value. The
former simply jumps out of the function and makes it return the value,
the latter specifies the value of the block which the function then
returns. The result is pretty much the same.
\begin{lstlisting}[language=vortex]
fn foo() -> do
    ...
end
\end{lstlisting}
The arrow feels kinda superfluous. Vortex allows you to omit it:
\begin{lstlisting}[language=vortex]
fn foo() do
    ...
end
\end{lstlisting}
Function arguments in Vortex can have default values. After you specify
the first default value, every argument after that one must specify a default
value. Example:
\begin{lstlisting}[language=vortex]
// you can reference the previous arguments too
fn foo(a, b = 5, c = b + 2, d = c + 3) do
    ...
end
\end{lstlisting}
You can end the argument list with an ellipsis argument. That means the
function will be variadic and you can access the remaining arguments
passed to the function again using an ellipsis. Example:
\begin{lstlisting}[language=vortex]
fn printf(fmt, ...) do
    print(fmt:format(...))
end
\end{lstlisting}
Note that to pass all the variadic arguments to a function, the ellipsis
must be the last argument of the function call. Otherwise just the first
value of the tuple will be passed! You can make it pass a single value
anywhere by capturing it in \verb|let|, for example
\begin{lstlisting}[language=vortex]
print("Only the first value", let _ = (...))
\end{lstlisting}
Lua allows you to do this by wrapping the ellipsis in parens. I consider
that quite dangerous and bug-prone, thus Vortex doesn't allow this. While
wrapping the ellipsis in parens in legal, it'll always evaluate to multiple
values, no matter what.\\
You can name the vararg tuple and pass it around as a table like this:
\begin{lstlisting}[language=vortex]
fn foo(a, b...) do
    print(a, b[1], b[2])
end
\end{lstlisting}
That is functionally equivalent to:
\begin{lstlisting}[language=vortex]
fn foo(a, ...) do
    let b = { ... }
    print(a, b[1], b[2])
end
\end{lstlisting}
\subsection{If expression}
\begin{bnf}
<ifblock> ::= \lit{do} <chunk>

<elseopt> ::= \lit{else} [ \lit{->} ] exp

<elsestatopt> ::= \lit{else} [ \lit{->} ] statexp

<ifexp> ::= \lit{if} <exp> (\lit{->} <exp> [ <elseopt> ] |
    <ifblock> (\lit{end} | <elseopt>))

<ifstat> ::= \lit{if} <exp> (\lit{->} <statexp> [ <elsestatopt> ] |
    <ifblock> (\lit{end} | <elsestatopt>))

<statexp> ::= <ifstat>

<exp> ::= <ifexp>
\end{bnf}
The \verb|if| expression allows you to do structured programming by
incorporating conditionals. Every \verb|if| expression begins with the
keyword, followed by a condition. The condition evaluates to either \verb|true|
or \verb|false|. If it evaluates to \verb|true|, it either evaluates to the
expression that follows the condition (when in expression form) or simply
executes the expression in statement form (in that case, only statement
form expressions are allowed).\\
There can be an optional \verb|else| part. That part is evaluated when the
condition is not met and has the same semantics as the former. Note that the
arrow is optional with \verb|else|, even with regular expressions.\\
When using \verb|if| with blocks and without arrows, the \verb|else| keyword
implicitly terminates the block scope. Using \verb|end| explicitly ends the
\verb|if| expression.
\subsection{Loops}
\begin{bnf}
<loopcond> ::= \lit{while} <exp>

<loopexp> ::= \lit{loop} [ <loopcond> ] <statscope> [ <loopcond> ]

<numforexp> ::= \lit{for} <ident> \lit{=} <exp> \lit{..} <exp>
    [ \lit{,} <exp> ] <statscope>

<genforexp> ::= \lit{for} <identlist> \lit{in} <explist> <statscope>

<statexp> ::= <loopexp>
    \alt <numforexp>
    \alt <genforexp>

<exp> ::= \lit{break}
    \alt \lit{cycle}
    \alt \lit{redo}
\end{bnf}
Vortex has three types of loops. The simplest one is the \verb|loop| loop.
By default, it loops like this:
\begin{lstlisting}[language=vortex]
loop do
    print("I'm infinite!")
end
\end{lstlisting}
You can add two kinds of conditions to this kind of loop, a regular condition
and a postcondition. Both work similarly, but with the postcondition the loop
iterates at least once before evaluating the condition (it evaluates AFTER
iteration) while with the regular condition it may never start (it evaluates
BEFORE iteration). You can use both at once. Example:
\begin{lstlisting}[language=vortex]
let i = 1
let keep_iteratng = false

fn check_iterate() -> true

loop while i <= 10 do
    print("I'm no longer infinite...")
    i += 1
    keep_iterating = check_iterate()
end while keep_iterating
\end{lstlisting}
Then there is the numeric \verb|for| loop. It iterates using a numeric range.
Both the start and the end of the range are inclusive. For example:
\begin{lstlisting}[language=vortex]
for i = 1 .. 10 do
    print(i)
end
\end{lstlisting}
This prints numbers from one to ten. There is an optional third \endquote{step}
expression. By default it's 1.
\begin{lstlisting}[language=vortex]
// 0, 2, 4, 6, 8, 10
for i = 0 .. 10, 2 do
    print(i)
end
\end{lstlisting}
The loop never re-evaluates the input expressions -- they're evaluated once
when the loop starts. The \verb|step| is particularly useful for backwards
iteration. Writing
\begin{lstlisting}[language=vortex]
for i = start, stop, step -> statexp
\end{lstlisting}
is equivalent to
\begin{lstlisting}[language=vortex]
do
    let (start, stop, step) =
        (tonum(start), tonum(stop), tonum(step or 1))
    if not (start and stop and step) -> error()
    loop while (step > 0 and start <= stop)
    or (step <= 0 and start >= stop) do
        let v = start
        statexp
        start += step
    end
end
\end{lstlisting}
Finally, there is the generic \verb|for| loop. It uses iterators and is
compatible with Lua iterators. You have a list of identifiers (you can have
as many as you want as long as the iterator handles them all). Then you have
the expression list, which is typically a single expression (an iterator).
Each iteration the iterator is called, returning a new set of values mapping
to the inputs. Writing
\begin{lstlisting}[language=vortex]
for k, v in pairs(tbl) -> statexp
\end{lstlisting}
is equialent to
\begin{lstlisting}[language=vortex]
do
    let (fun, s, var) = pairs(tbl)
    loop do
        let (k, v) = fun(s, var)
        if k == nil -> break
        var = k
        statexp
    end
end
\end{lstlisting}
All types of loops accept both arrow notation and blocks, but even in arrow
notation and expression form the expression past the arrow ALWAYS must be
a statement. A loop is an expression, but it evaluates to \verb|nil|.\\
You can control the loop using three expressions, \verb|break|, \verb|cycle|
and \verb|redo|. Using \verb|break| you can stop the iteration at that point.
Using \verb|cycle| you can skip to the next iteration, incrementing counters
or calling iterator as needed. Using \verb|redo| you can achieve a similar
thing, but no counter is ever incremented (or iterator called), effectively
restarting the current iteration. The \verb|cycle| keyword is equivalent to
\verb|continue| in several other languages (such as C, C++ or JavaScript).
\subsection{Pattern matching}
\begin{bnf}
<objectpatitem> ::= ([ \lit{\$} ] <ident> | \lit{\$(} <exp> \lit{)})
    [ \lit{:} ([ \lit{\$} ] <ident> | \lit{\$(} <exp> \lit{)}) ]

<objectpatbody> ::= <objectpatitem> \{ \lit{,} <objectpatitem> \}

<tablepatitem> ::= [ ([ \lit{\$} ] <ident> | \lit{\$(} <exp> \lit{)})
    \lit{:} ] <pattern>

<primarypattern> ::= \lit{(} <pattern> \lit{)}
    \alt <strliteral>
    \alt <numliteral>
    \alt \lit{true}
    \alt \lit{false}
    \alt \lit{nil}
    \alt \lit{_}
    \alt [ \lit{\$} ] <ident> [ \lit{(} [ <objectpatbody> ] \lit{)} ]
    \alt \lit{\$(} <exp> \lit{)} [ \lit{(} [ <objectpatbody> ] \lit{)} ]
    \alt \lit{\{} <tablepatitem> \{ \lit{,} <tablepatitem> \} \lit{\}}

<suffixedpattern> ::= <primarypattern>
    \alt <suffixedpattern> \lit{when} <exp>
    \alt <suffixedpattern> \lit{as} <exp>

<patternop> ::= \lit{and}
    \alt \lit{or}
    \alt \lit{::}

<pattern> ::= <suffixedpattern> [ <patternop> <suffixedpattern> ]

<matchexp> ::= \lit{match} <explist> \lit{->} <matchbody>

<matchstat> ::= \lit{match} <explist> \lit{->} <matchbodystat>

<matcharm> ::= (\lit{|} | \lit{case}) <patternlist> <expscope>

<matcharmstat> ::= (\lit{|} | \lit{case}) <patternlist> <statscope>

<matchbody> ::= <matcharm> \{ <matcharm> \}

<matchbodystat> ::= <matcharmstat> \{ <matcharmstat> \}

<exp> ::= <matchexp>

<statexp> ::= <matchstat>
\end{bnf}
Vortex provides pattern matching similarly to e.g. the MLs, Rust or Haskell.
Pattern matching works like a generalized \verb|switch| statement -- you have
a list of input expressions in the \verb|match| expression and then you have
some arms - arms start either with \verb|\|| or with \verb|case|, followed
by a list of patterns (one pattern for each input, if you don't provide
a pattern for some input then it always matches the input), then followed
by either an expression after an arrow or a \verb|do| block.\\
The same rules as with e.g. \verb|if| apply -- when \verb|match| is used in
a statement form, only statement form expressions are allowed.\\
The arms are evaluated from the top. The first arm where all patterns match
is evaluated and then the evaluation stops. As you can see, it's similar to
a \verb|switch|, but there is no fallthrough. Also unlike for example
OCaml, non-exhaustive patterns are NOT detected.\\
Pattern matching can be used for decomposition of data structures, besides
regular mathching. For that purpose, lots of patterns are provided.
\begin{description}
\item[Variable pattern] \hfill \\
This is the simplest kind of pattern. It's just a name (an identifier) and
it is used to capture the input into a variable (local to the specific arm).
You can then use the input using that variable. It never fails to match.
\item[Wildcard pattern] \hfill \\
This one is similar to variable pattern, except that no variable capture is
made (it also never fails to match). It's an underscore in code.
\item[Expression pattern] \hfill \\
This pattern is represented either as a string literal, a number literal,
\verb|true|, \verb|false|, \verb|nil|, an identifier prefixed by \verb|$|
or an expression enclosed in \verb|$()|. It doesn't capture and it may
fail to match (it tests the input for equality with the given expression).
\item[Table pattern] \hfill \\
This pattern matches tables. It looks like a table literal. It can match
both array and hash members, where array members are written in the same
way as expressions in an array literal (except that they're patterns) and
hash members are written in the same as well, where the key is an expression
and the value is a pattern. When matching a table, you have to match all
the array members, but you don't have to match all the hash members.
It may fail to match.
\item[Object pattern] \hfill \\
With this pattern you can match objects. It consists of an identifier or
a \verb|$| or a \verb|$()| expression followed by parens that contain a
list of object member captures. A capture can be a simple identifier (then
it will capture a member of that name) or a \verb|$| or a \verb|$()| expression
followed by a colon and then by an identifier -- in that case it'll match
a member with the key the expression evaluates to into a variable after the
colon. You don't have to capture all object members. It may fail to match.
\item[Cons pattern] \hfill \\
This binary pattern decomposes an input into two parts, head and tail. It has
a reversed meaning to the cons operator. The input has to be either a table
or a list. On a table this is slow, on a list this is fast (in case of table
it has to actuall slice the table). On invalid input it fails to match.\\
Binary patterns have their precedence and associativity. The cons pattern has
the highest precedence and is right associative. It looks the same as the cons
operator.
\item[And, or patterns] \hfill \\
Binary patterns. The \verb|or| pattern has the lowest precedence, the
\verb|and| pattern is above it, followed by the cons pattern. These are all
binary patterns in Vortex. Both \verb|and| and \verb|or| patterns are left
associative. They may fail to match.
\item[When pattern] \hfill \\
This pattern is a conditional pattern. It consists of a pattern followed by
the keyword \verb|when| followed by a condition (which is an expression).
It may obviously fail to match.
\item[As pattern] \hfill \\
This pattern essentially captures a pattern. It consists of a pattern followed
by the \verb|as| keyword followed by another pattern. It's particularly useful
if you have for example an expression pattern that doesn't capture the input
and you still want to capture it besides checking its equality with the
expression.
\end{description}
Some example code for pattern matching:
\begin{lstlisting}[language=vortex]
let x = 5
// prints 5 - variable pattern
print(match x -> | y -> y)

// prints 10, expression and wildcard patterns
print(match 4 ->
    | 1 -> 2
    | 2 -> 3
    | 3 -> 4
    | _ -> 10)

// prints 5
print(match 4 ->
    | $(2 + 3) as x -> x
    | $(2 * 2) as y -> y + 1)

// prints nil, doesn't match
print(match 10 ->
    | x when x == 5 -> x)

let tbl = { 5, 10, 15, foo: "bar", bar: "baz" }
// prints "baz", first arm is incomplete
print(match tbl ->
    | { a, b, foo: c } -> a
    | { a, b, c, bar: d } -> d)

let list = [ 5, 10, 15, 20 ]
// reversed list
let x = match list ->
    | a :: b :: c :: d -> [ d, c, b, a ] )
\end{lstlisting}
As previously mentioned, the \verb|let| expression also makes use of patterns.
The pattern use is limited though -- only patterns that never fail to match
can be used. The table pattern is modified appropriately to allow use with
\verb|let| -- you can match partial contents of an array with it.
\subsection{Objects}
\begin{bnf}
<objectparents> ::= <suffixedexp>
    \alt \lit{(} <explist> \lit{)}

<objectimplctor> ::= \lit{[} [ <identlist> ] \lit{]}

<objectitem> ::= <fndef>
    \alt ([ \lit{\$} ] <ident> | \lit{\$(} <exp> \lit{)}) \lit{:} <exp>

<objectbody> ::= <objectimplctor> [ \lit{do} \{ <objectitem> \} \lit{end} ]
    \alt \lit{do} \{ <objectitem> \} \lit{end}

<objectexp> ::= \lit{clone} [ <objectparents> ] <objectbody>

<newexp> ::= \lit{new} (<primaryexp> | \lit{(} <explist> \lit{)})

<exp> ::= <objectexp>
    \alt <newexp>
\end{bnf}
Vortex features a builtin object system. It's prototype based, delegative
and supports multiple inheritance. It also has operator overloading. The object
system is based around the \verb|clone| expression. It clones a parent object
(or a set of parent objects in case of multiple inheritance). If you don't
provide a parent object, it inherits from the internal Object table (which
provides the basic stuff concerning objects). That means Object is the base
for every user defined object.\\
When inheriting from multiple parents, non-existent members are looked up
from parents from left to right depth-first.\\
The \verb|clone| expression cannot be used as a statement. Objects are first
class values and the \verb|clone| expression evaluates to them, so you use
\verb|clone| in combination with \verb|let|.\\
Objects feature constructors, but they don't have destructors. There are GC
finalizers provided in the same manner as the \verb|__gc| metamethod with
tables. There is an implicit constructor syntax -- you can provide a list
of identifiers in square brackets that represent member names and Nth
constructor argument will become Nth member in the brackets.\\
Constructors are never called when cloning an object.\\
There is also the \verb|new| expression, which creates an \enquote{instance}
of the object -- that means, it clones the object and calls the parent
constructor on it. That allows you to pretty much closely emulate classes
without losing the flexible prototypal nature of Vortex's object system.\\
Vortex provides the \verb|super| function in the standard library. It returns
a proxy object on which one can call parent methods. You can provide either
one or two arguments. If you have an object \verb|x|, which is an instance of
\verb|Bar|, which is a clone of \verb|Foo|, calling \verb|super(x):abc()|
calls a method of \verb|Foo| (instead of \verb|Bar|, which is a parent of
\verb|x|) and it clals it on \verb|x|. If you provide two arguments, the
first one is an object -- a clone, the other one is the object on which
we want to call. Calling \verb|super(Bar, x)| and \verb|super(x)| is equivalent
in this case.
Some examples:
\begin{lstlisting}[language=vortex]
let Foo = clone do
    fn __init(self, a, b) do
        print("I'm a constructor!")
        set (self.a, self.b) = (a, b)
    end

    fn foo() do
        print("I'm a method without self, all alone")
    end

    fn bar(self) -> print("I'm a method!", self.a, self.b)
end

let Bar = clone Foo do
    fn xyz(self) -> Foo.bar(self)
end

let Baz = clone [ a, b, c ]

let inst = new Baz(5, 10, 15)
print(inst.a, inst.b, inst.c)
\end{lstlisting}
\subsection{Other expressions}
\begin{bnf}
<exp> ::= \lit{__FILE__}
    \alt \lit{__LINE__}
\end{bnf}
There are two other expressions. They're very simple. The \verb|__FILE__|
expression expands to the current filename at compile time. The \verb|__LINE__|
expression expands to the line the expression is on, also at compile time.

\section{Macros}
\begin{bnf}
<macro> ::= \lit{macro} <ident> [ <identlist> ] [ \lit{...} ] <expscope>

<quoteexp> ::= \lit{quote} (<suffixedexp> | <block>)

<unquoteexp> ::= \lit{unquote} (<suffixedexp> | <block>)

<exp> ::= <quoteexp>
    \alt <unquoteexp>
\end{bnf}
Vortex features an AST based macro system. That means it operates on AST level
rather than on text level -- that way it can be context aware and much safer.
The currently implemented macro system is not hygienic (it can capture and
create outer identifiers) but it's planned.\\
A macro starts with the \verb|macro| keyword followed by the macro name
and an argument list. Then either an arrow followed by an expression or
a block follows. This is very similar to regular functions, but there are
no default argument values and no named varargs.\\
To use a macro, you need to expand it. You do that using the
\verb|macroname!(macroargs)| syntax (it's a primary expression). It means
that the macro will expand at that point, substituting the expansion expression
with some actual code.\\
As mentioned previously, a macro works on AST level -- the macro has zero or
more inputs (that are serialized AST nodes) and it's supposed to return an
AST node, which is then injected into the final AST at the expansion point.\\
Macro arguments are implicitly serialized into AST nodes. The expression you
return from the macro (using regular \verb|return| or using the arrow notation)
is not, so you need to serialize it. For that purpose, there is the
\verb|quote| expression. It takes an arbitrary expression and evaluates to
the AST of it, converting any subexpressions into AST as well. For example
\begin{lstlisting}[language=vortex]
let x = quote (fn a, b do
    print("hello world")
end)
\end{lstlisting}
results in a function AST node with all the contents serialized as well as
the function itself.\\
There is one other expression, \verb|unquote|. It basically tells the
\verb|quote| expression to \enquote{stop} at that point. For example:
\begin{lstlisting}[language=vortex]
let a = quote (x + y)
let b = quote (y * z)

// this is a binary expression with
// operator / containing two symbols
let c = quote (a / b)
// this is a binary expression with operator
// / containing two binary expressions, a and b
let d = quote (unquote a / unquote b)
\end{lstlisting}
With these facilities available, we can define a macro:
\begin{lstlisting}[language=vortex]
macro my_if(cond, texp, fexp) -> quote
    (match not unquote cond ->
        | true  -> unquote fexp
        | false -> unquote texp)

let x = 5
print(my_if!(x == 5, "hello", "world"))
\end{lstlisting}
Here it prints \enquote{hello}, because the condition is true. The unquotes
are needed so that we put the inputs themselves in the result, not just
symbols.

\section{The REPL}
Vortex provides an interactive command line and a standalone script runner
combined into a single script. By default it launches an interactive session.
There you can input statements.\\
Local variables are preserved.
\begin{lstlisting}[language=vortex,numbers=none]
> let x = { 5, 10, { 15, 20 } }
> =x // proper output
{ 5, 10, { 15, 20 } }
\end{lstlisting}
The Vortex REPL in the default implementation is itself written in Vortex.
It requires a Lua module called \verb|vxutil|. This module is shipped with
Vortex and is written in C -- you need to compile it and then put into a
Lua C module search path (for example the current directory). The C module
provides support for signals, \verb|isatty| and the readline library. All
functions have their fallbacks. For example, if you want readline support in
the REPL (so that you can move in the history and seek in the current input),
you need to compile the C module with \verb|VX_READLINE| defined. For proper
function of \verb|isatty| you need to compile the module with either
\verb|VX_POSIX| or \verb|VX_WIN| defined depending on your platform. If you
don't define either of these, a fallback will always return true, which
means autodetection whether to launch an interactive session won't work
(when \verb|isatty| returns false, it means something is piped into the
standard input, and the REPL tries to execute that).
\begin{lstlisting}[language=text]
usage: vortex [options] [script [args]]
-e str  run the string 'str'
-i      enter interactive mode after the options are handled
-l lib  require library 'lib'
-v      show version information
--      stop handling options
-       stop handling options and run stdin
\end{lstlisting}
Here you can see the REPL options you can pass to it. As mentioned above,
you can use the REPL with pipes. For example
\begin{lstlisting}[language=text]
echo 'print("hello world")'|<replcommand>
\end{lstlisting}
should work.\\
If your interactive line starts with \verb|=| followed by an expression,
the REPL prints its value. You can see an example in the very beginning
of this section. Tables get special treatment -- they're serialized before
printing. That should allow you to see the values clearly.

\section{Appendix: Style guide}
These are recommended conventions for Vortex code. You don't have to follow
them, but it's strongly recommended for consistency.
\subsection{Indentation}
You use 4 spaces to indent each level. Tabs should be avoided. Try not to
follow the Lua convention of 2 spaces.
\subsection{Whitespace}
Put a space before and after a binary operator. Do not space off unary
operators. Do not put spaces around parens. You should put a space after each
comma in the language. Do not put a space between function name and its
argument list. You can use spaces to align things where needed.
\subsection{Blocks}
Keep the \verb|do| keyword on the same line with the preceeding code.
\subsection{Naming style}
The Vortex naming rules are simple and encourage readable code.
\begin{itemize}
\item Values, modules, functions and variables generally are \verb|snake_case|.
\item Objects are \verb|This_Case|.
\item Macro names are \verb|snake_case|. No need for uppercase because of
obvious expansion syntax.
\end{itemize}
\begin{lstlisting}[language=vortex]
let foo = 5 // Good
let Bar = 10 // Bad!

fn foo_bar(a, b) -> expr() // Good
fn fooBar(a, b) -> expr() // Bad

// Good
let My_Object = clone do
    fn my_method(self) -> self.x
end
// Bad
let MyObject = clone do
    fn myMethod(self) -> self.x
end

// Good
macro foo(a, b) do ... end
print(foo!(5, 10))
// Bad
macro FOO(a, b) do ... end
print(FOO!(5, 10))
\end{lstlisting}
\subsection{Examples}
\begin{lstlisting}[language=vortex]
// Good!
let x = a + b

// Also good.
if foo do
    print("boo!")
end

// Alignment is good
foo(bar(),
    baz())

// Spacing makes things readable.
foo(bar(), baz(), (5 + 10 * (34 - 1) / 150))

// Functions, the good way
fn foo(a, b) -> hello(a, b)

// But this is bad.
let x = a+b

// This is bad too.
if foo
do
    print("boo")
end

// Bad alignment
foo(bar(),
        baz())

// Excessive spacing.
foo ( bar (), baz (), ( 5 + 10 * ( 34 - 1 ) / 150 ) )

// A badly written function
fn foo (a, b) -> hello ( a, b )
\end{lstlisting}
\section{Appendix: Influences}
Vortex is a language with many influences. It follows the same principles as
Lua, on which is built. Thus, you can find many similarities in Lua's and
Vortex's syntax and semantics as well as the core set of features. Vortex and
Lua are both built around tables, the ultimate do-it-all data structure.
The reliance on higher order and first class functions is also taken very
seriously in both languages.\\
The second largest influence for Vortex is OCaml. Vortex takes many primarily
syntactic features from OCaml, including the pattern matching syntax. Vortex
is however, unlike OCaml, an untyped language. The features are thus adjusted
accordingly for Vortex.\\
OCaml's relative, F\#, is also influental for Vortex. The basic idea of
sequences in place of list comprehensions is taken from F\#, but unlike
F\# sequences are not values in Vortex, instead they're simply tuples
of values that can't be treated in a first class manner.\\
The prototype based multiple inheritance object system of Vortex is inspired
by Io. Compared to Io, the constructors work differently and Vortex's object
system allows you to pretty much closely imitate classes (those are handled
a bit differently in Io).\\
Scheme inspired Vortex in how simple a language can be while remaining very
powerful. Along with Elixir, Scheme provided a basis for Vortex AST macros.\\
Elixir and Ruby influenced Vortex's syntax. The Rust language provided an
example how terse a syntax can be and several of Vortex's keywords are or
were taken from Rust. The \verb|cycle| keyword is taken from Fortran.
String literal syntax is taken from Python.\\
Other languages that influenced Vortex to some degree are ALGOL, C, CLU,
Haskell and Self.

\end{document}
